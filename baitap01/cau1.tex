\textbf{Bài tập 1:}
Hãy tìm kiếm và nêu ra 2 ví dụ ứng dụng của tối ưu hóa.
Chú ý: Cần nêu cụ thể chi tiết, không chỉ nêu tên.

\textbf{VD1:} 

Tối ưu hóa quy trình sản xuất:

Trong quản lý sản xuất, tối ưu hóa được 
sử dụng để cải thiện quy trình sản xuất và giảm thiểu chi phí sản xuất.
 Một ví dụ cụ thể là tối ưu hóa tỷ lệ pha trộn cho một sản phẩm hóa học, 
 trong đó hỗn hợp pha trộn bao gồm các thành phần khác nhau với chi phí 
 khác nhau. Mục tiêu của tối ưu hóa tỷ lệ pha trộn là tối thiểu hóa chi
phí sản xuất và đảm bảo chất lượng sản phẩm đạt tiêu chuẩn.

Giả sử, ta có một hỗn hợp gồm các thành phần A, B và C với giá
 cả lần lượt là 10, 20, 30. 
 Một lô sản phẩm phải có độ tinh khiết tối thiểu của thành phần A là $50\%$
  và độ tinh khiết tối thiểu của thành phần B là $30\%$.
 Ngoài ra, tổng số lượng hỗn hợp pha trộn phải đạt mức tối 
 đa là $100$ đơn vị.

%gõ kí tự $

Phương trình tối ưu hóa tỷ lệ pha trộn có thể được viết như sau:
Gọi x, y và z lần lượt là tỷ lệ pha trộn của các thành phần A, B và C.
 Giải phương trình tối ưu hóa này, ta có thể tìm được
  tỷ lệ pha trộn tối ưu cho mỗi thành phần, giúp cải thiện quy trình sản xuất và giảm thiểu chi phí sản xuất.

minimize: $10x + 20y + 30z$ (chi phí sản xuất)\\
sao cho:\\
$x + y + z <= 100 $(giới hạn số lượng sản phẩm)\\
$x >= 0.5(x+y+z)$ (độ tinh khiết thành phần A)\\
$y >= 0.3(x+y+z)$ (độ tinh khiết thành phần B)\\
$x,y,z >= 0 $\\
$x,y,z \in \mathbb{Z}$

(Trong thực tế, độ tinh khiết của các chất thường ở mức $99\%$.)

\textbf{VD2:} 

Tối ưu hóa trong đua xe 
Trong đua xe, tối ưu hóa được sử dụng để tìm ra cách tối ưu hóa hiệu suất và tốc độ của xe trên đường đua. Mỗi loại xe có các thông số về động cơ, trọng lượng, lực nâng và gia tốc tối đa khác nhau. Tối ưu hóa được sử dụng để tìm ra các thông số tối ưu cho xe để đạt được tốc độ cao nhất trên đường đua.

Ví dụ, trong một bài đua xe địa hình, xe phải vượt qua nhiều chướng ngại vật như đồi, hố sâu, đá và những đoạn đường khó khăn. Để tối ưu hóa hiệu suất và tốc độ của xe, tối ưu hóa được sử dụng để tìm ra các thông số tối ưu cho xe, bao gồm lực nâng, tốc độ, tăng tốc và hệ số ma sát của bánh xe.

Phương trình toán học được sử dụng trong tối ưu hóa đua xe bao gồm phương trình vật lý như phương trình chuyển động của Newton và phương trình vật lý địa hình như phương trình địa hình của Baldwin và Lomax. Ngoài ra, các thuật toán tối ưu hóa như gradient descent, stochastic gradient descent và thuật toán tối ưu hóa di truyền
 có thể được sử dụng để tìm ra các thông số tối ưu cho xe.
