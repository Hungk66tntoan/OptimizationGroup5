%Bài tập 3. (bài tập mô hình hóa - mở đầu)
%Nhóm 5
%Bài 3. (bài tập mô hình hóa - mở đầu)
\begin{exercise}{3}{Bài tập mô hình hóa - mở đầu}

    Xét một bài toán nổi tiếng:
    %lùi 3cm

    \begin{problem}{10}
        Đặt nhiều quân hậu trên bàn cờ vua tiêu chuẩn 8 × 8 nhất có thể sao cho
không có hai quân hậu nào tấn công nhau.
    \end{problem}
    \begin{solution}
    Bàn cờ vua tiêu chuẩn có 8 hàng và 8 cột. Ta định nghĩa một đường chéo là một tập hợp gồm ít
    nhất hai tọa độ, các tọa độ cùng nằm trên một đường thẳng tọa với trục ngang (cũng như trục
    dọc) một góc $45$ độ. Có tất cả 26 đường chéo trên bàn cờ vua. Một quân hậu có thể tấn công
    một quân khác trên một ô cùng hàng ngang, cột dọc, hoặc đường chéo nếu những ô ở giữa là trống.

    Ta sẽ xây dựng một mô hình tối ưu tuyến tính nguyên giải quyết bài toán trên.

    Đầu tiên, cần tìm cách biểu diễn vị trí đặt những quân hậu. Với mỗi tọa độ (i, j) trên bàn cờ,
ta định nghĩa một biến $x_{i,j}$ nhận giá trị lần lượt là 1 hoặc 0 ứng với trường hợp tọa độ đó có
quân hậu hoặc không có quân hậu.
%liệt kê a,b,c
%a)
%b)
%c)
    \be
\begin{enumerate}

    \item ádf
    \item à
\end{enumerate}













\end{solution}
\end{exercise}