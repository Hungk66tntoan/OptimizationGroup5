\documentclass[12pt]{article}
\usepackage[utf8]{vietnam}
\usepackage{amsmath}
\usepackage{amsfonts}
\usepackage{amssymb}
\usepackage{amsopn}
\usepackage{amsthm}
\usepackage{commath}
\usepackage{graphicx}
\usepackage{subcaption}
\usepackage{amsxtra}
\usepackage{indentfirst}
\usepackage{pdfpages}
\usepackage[sort&compress]{natbib}
\usepackage{wasysym}
\usepackage{mathtools}
\usepackage{physics}
\usepackage[mathscr]{eucal}

\DeclareMathOperator{\spn}{span}
\DeclareMathOperator{\dis}{d}
\newcommand{\R}{\mathbb{R}}
\newcommand{\N}{\mathbb{N}}
\newcommand{\K}{\mathbb{K}}
\newcommand{\C}{\mathbb{C}}
\newcommand{\codim}{\operatorname{codim}}
\newcommand{\olsi}[1]{\,\overline{\!{#1}}}
\newtheorem{lemma*}{Bổ đề}
\newtheorem{theorem}{Định lý}
\newtheorem{proposition}{Bài}
\newtheorem{corollary*}{Chứng minh}
\newenvironment{solution}{%
     \setlength\parindent{0pt}\par\medskip\textbf{Lời giải.}\quad}{%
     \hfill\tiny$\blacksquare$\par\medskip}
\usepackage[left=2cm,right=2cm,top=2cm,bottom=2cm]{geometry}
\setlength{\parindent}{0pt}
\begin{document}
    Thông tin các thành viên trong nhóm 5:
    \begin{enumerate}
        \item[1.] Lê Thị Thu An, 18001975, K63 TN Toán học.
        \item[2.] Thiều Đình Minh Hùng, 21000006, K66 TN Toán học.
    \end{enumerate}
    \begin{center}
        $*********$
    \end{center}
    \textbf{Bài tập 1.} Xét hàm số Rosenbrock:
    \begin{align*}
        f(x_1,x_2) = 100(x_2 - x_1^2)^2 + (1 - x_1)^2
    \end{align*}
    \begin{enumerate}
        \item[1.] Hàm $f$ có phải là một hàm lồi hay không? Tại sao?
        \item[2.] Hãy tìm tất cả các điểm cực tiểu địa phương của $f$. Hàm $f$ có điểm cực tiểu toàn cục không?
        Nếu có hãy chỉ ra đó là điểm nào?
    \end{enumerate}
    \begin{solution}
        \begin{enumerate}
            \item[1.] Để ý rằng $f \in C^{\infty}(\R^2)$ và:
            \begin{align*}
                \begin{cases}
                    \pdv{f}{x_1}(x_1,x_2) = -400x_1x_2 + 400x_1^3 + 2x_1 - 2, \pdv{f}{x_2}(x_1,x_2) = 200(x_2 - x_1^2)\\
                    \frac{\partial^2 f}{\partial x_1^2}(x_1,x_2) = -400x_2 + 1200x_1^2 + 2, \frac{\partial^2 f}{\partial x_2^2}(x_1,x_2) = 200\\
                    \frac{\partial^2 f}{\partial x_1 \partial x_2}(x_1,x_2) = \frac{\partial^2 f}{\partial x_2 \partial x_1}(x_1,x_2) = -400x_1
                \end{cases}
            \end{align*}
            Ma trận Hessian của $f$ tại $(x_1,x_2)$ là:
            \begin{align*}
                H_f(x_1,x_2) = \begin{pmatrix}
                    -400x_2 + 1200x_1^2 + 2 & -400x_1\\
                    -400x_1 & 200
                \end{pmatrix}
            \end{align*}
            Chú ý rằng $\det(H_f(x_1,x_2)) = 200(1200x_1^2 - 400x_2 + 1) - (-400x_1)^2 = 200(4000x_1^2 - 400x_2 + 1)$. Như vậy $\det(H_f(0, 1)) = -79800 < 0$ và do đó $H_f(0, 1)$ không xác định dấu. Suy ra $f$ không là một hàm lồi.
            \item[2.] Ta có:
            \begin{align*}
                \begin{cases}
                    \pdv{f}{x_1}(x_1,x_2) = 0\\
                    \pdv{f}{x_2}(x_1,x_2) = 0
                \end{cases} &\Leftrightarrow \begin{cases}
                    -400x_1x_2 + 400x_1^3 + 2x_1 - 2 = 0\\
                    200(x_2 - x_1^2) = 0
                \end{cases}\\ &\Leftrightarrow \begin{cases}
                    x_2 = x_1^2\\
                    2x_1 - 2 = 0
                \end{cases} \Leftrightarrow x_1 = x_2 = 1
            \end{align*}
            Hơn nữa ma trận Hessian của $f$ tại $(1,1)$ là:
            \begin{align*}
                H_f(1,1) = \begin{pmatrix}
                    802 & -400\\
                    -400 & 200
                \end{pmatrix},
            \end{align*}
            đồng thời dễ thấy ma trận này xác định dương do $\det(H_f(1,1)) = 400 > 0$ và $802 > 0$. Vì vậy, $(1,1)$ là một điểm cực tiểu địa phương của $f$.
            \\
            Để ý rằng $f(1, 1) = 0$, đồng thời dễ thấy $f(x_1,x_2) \geq 0$ với mọi $(x_1,x_2) \in \R^2$. Vậy $f$ có điểm cực tiểu toàn cục là $(1,1)$.
        \end{enumerate}
    \end{solution}
\end{document}