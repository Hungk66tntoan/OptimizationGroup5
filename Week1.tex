\documentclass[12pt, draft]{article}
\usepackage[utf8]{vietnam}

\usepackage{amsmath}
\usepackage{amsfonts}
\usepackage{amssymb}
\usepackage{amsthm}
\newtheorem{lemma}{Bổ đề}
\newtheorem{theorem}{Định lý}
\newtheorem{proposition}{Bài}
\newenvironment{solution}{%
     \setlength\parindent{0pt}\par\medskip\textbf{Lời giải.}\quad}{%
     \hfill\tiny$\blacksquare$\par\medskip}
\usepackage[left=2cm,right=2cm,top=2cm,bottom=2cm]{geometry}
\setlength{\parindent}{0pt}



\begin{document}

Bài tập 2: Hãy xây dựng mô hình quy hoạch tuyến tính của câu a bài X trang 169-170 trong tài liệu sau 
\\
Lưu ý, chỉ cần xây dựng mô hình, chưa cần giải đưa ra kết quả.
\\
Phân tich bài toán:
\\
Bước 1: Xác định các biến.
\\
$x_1$ - khoản đầu tư vào cổ phiếu EAL;
\\
$x_2$ - khoản đầu tư vào cổ phiếu BRU;
\\
$x_3$ - khoản đầu tư vào cổ phiếu TAT;
\\
$x_4$ - khoản đầu tư vào trái phiếu ngắn hạn;
\\
$x_5$ - khoản đầu tư vào trái phiếu dài hạn;
\\
$x_6$ - khoản đầu tư vào niên kim hoãn thuế.
\\
(đơn vị là dollar)
\\
Bước 2: Xác định hàm mục tiêu
\\
Hàm mục tiêu ở đây chính là lợi nhuận hàng năm thu về, hàm này cần đạt giá trị lớn nhất:
\begin{center}
    $F(x_1, x_2, x_3, x_4, x_5, x_6) = 0.15 x_1 + 0.12 x_2 + 0.09 x_3 + 0.11 x_4 + 0.08 x_5 + 0.06 x_6$
\end{center}
Bước 3: Xác định các điều kiện hiện và điều kiện ẩn.
\\
- Vì toàn bộ \text{\$50000} được đầu tư nên:
\begin{center}
    $x_1 + x_2 + x_3 + x_4 + x_5 + x_6 = 50000$
\end{center}
- Vì ít nhất \$10000 được đầu tư vào niên kim hoãn thuế nên:
\begin{center}
    $x_6 \geq 10000$
\end{center}
- Vì ít nhất 25\% vốn đầu tư vào cổ phiếu tập trung vào cổ phiếu TAT nên:
\begin{center}
    $x_3 \geq \frac{1}{4} (x_1 + x_2 + x_3),$
\end{center}
tức là:
\begin{center}
    $x_1 + x_2 - 3 x_3 \geq 0$
\end{center}
- Vì số vốn đầu tư vào trái phiếu ít nhất phải bằng số vốn đầu tư vào cổ phiếu nên:
\begin{center}
    $x_4 + x_5 - x_1 - x_2 - x_3 \geq 0$
\end{center}
- Vì số vốn đầu tư vào các nguồn đầu tư sinh lời dưới 10\% không quá \$12500 nên:
\begin{center}
    $x_3 + x_5 + x_6 \leq 12500$
\end{center}
- Ràng buộc ẩn: $x_1, x_2, x_3, x_4, x_5 \geq 0.$
\\
Kết quả mô hình thu được:
\begin{align}
    \\
    $\text{MAX   } 0.15 x_1 + 0.12 x_2 + 0.09 x_3 + 0.11 x_4 + 0.08 x_5 + 0.06 x_6$
    \\
    $\text{s.t.}$
    \\
    1) $x_1 + x_2 + x_3 + x_4 + x_5 + x_6 = 50000$
    \\
    2) $x_1 + x_2 - 3 x_3 \leq 0$
    \\
    3) $x_4 + x_5 - x_1 - x_2 - x_3 \geq 0$
    \\
    4) $x_3 + x_5 + x_6 \leq 12500$
    \\
    5) $x_1, x_2, x_3, x_4, x_5 \geq 0$
    \\
    6) $x_6 \geq 10000$
\end{align}
\\
Bài tập 3: 
\end{document}